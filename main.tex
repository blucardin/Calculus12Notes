\documentclass{article}

\usepackage[utf8]{inputenc} % set input encoding (not needed with XeLaTeX)
\usepackage[top=1cm,bottom=1cm,left=1cm,right=1cm,marginparwidth=1.75cm]{geometry} % set the margins

\usepackage{amsmath} % provides many mathematical environments & tools
\usepackage{amssymb} % provides many mathematical symbols

\usepackage{titling} % provides \thetitle, \theauthor, \thedate
\usepackage[export]{adjustbox} % provides \includegraphics[valign=c]{...}
\usepackage{wrapfig} % provides wrapfigure environment for figures with text wrapping
\usepackage{xcolor} % before tikz or tkz-euclide if necessary
\usepackage{soul} % provides \hl{} for highlighting, \st{} for strikethrough, \ul{} for underline

\usepackage[usestackEOL]{stackengine}[2013-10-15]
\stackMath
\usepackage{graphicx}
\setlength{\footskip}{15pt}
\setlength{\arrayrulewidth}{0.5mm}
\setlength{\tabcolsep}{18pt}
\renewcommand{\arraystretch}{2} 

\usepackage{siunitx} % provides units
\usepackage{cancel} % provides \cancel{} and \bcancel{} for crossing out in math mode
\usepackage{lmodern} % Latin Modern font, vectorized and with more glyphs

\usepackage[makeindex]{imakeidx} % provides indexing
\makeindex[columns=2, title=Index, intoc] % set up the index
\newcommand{\boldindex}[1]{\textbf{#1}\index{#1}} % a new command that bolds text and adds it to the index

\usepackage{draculatheme} % provides a dark mode for the document
% remember to change the \hypersetup colors if using this package

% \usepackage{darkmode} 
% \enabledarkmode

\newcommand{\code}{MCV4U}
\newcommand{\name}{Calculus and Vectors 12}
\title{\code\ Course Notes}
\author{Noah Virjee}
\date{June 2022}

\usepackage{hyperref}
% \hypersetup{ % set up hyperref to have a nice looking set of colors
%     colorlinks=true,
%     linkcolor=blue,
%     filecolor=magenta,      
%     urlcolor=blue,
%     pdftitle={\name},
%     pdfpagemode=FullScreen,
%     }

% if dracula theme is enabled, set the hyperref colors to match
\hypersetup{
    colorlinks=true,
    linkcolor=draculapurple,
    filecolor=draculapurple,      
    urlcolor=draculapurple,
    pdftitle={\name},
    pdfpagemode=FullScreen,
    }

\urlstyle{same} % set up the url style to be the same as the rest of the document

\begin{document}

\noindent\parbox{\linewidth}{ % title and author block
\parbox{.7\linewidth}{\fontsize{24}{28}\selectfont\thetitle}\hfill%
\parbox{.3\linewidth}{\fontsize{12}{14}\selectfont\raggedleft\today\\\theauthor%
}}

\begin{abstract}
This document is a review of the \name\ course. I had some time on my hands and I really needed to study for exams so here goes nothing. If you find any mistakes, please let me know so I can fix them. I hope this helps you!
\end{abstract}

\tableofcontents
\pagebreak

\section{Calculus}

\subsection{Limits}
The \boldindex{Limit} of a function is the value of a function as its input approaches some value. 
\begin{equation}
    \lim_{x\to a}f(x)
\end{equation}

The \boldindex{Right-hand Limit} is the value of a function as its input approaches some value from the positive side.
\begin{equation}
    \lim_{x\to a^{+}}f(x)
\end{equation}

The \boldindex{Left-hand Limit} is the value of a function as its input approaches some value from the negative side.
\begin{equation}
    \lim_{x\to a^{-}}f(x)
\end{equation}

The left and right hand limits only exist if there is no vertical asymptotes at $x$ and $f(x)$ exists at the x-values approaching $x$. 

The limit of a function at a point only exists if both the left and right limits exist and are equal to each other. 

A function is said to have \boldindex{Continuity} at a point if: 
\begin{equation}
    \lim_{x\to a}f(x) = f(x)
\end{equation}

\subsubsection{Limit Properties}

Suppose that $\lim_{x \to a} f(x)$ and $\lim_{x \to a} g(x)$ exist.
\begin{enumerate} 
    \item \begin{equation}\lim_{x \to a} c = c\ ,\ c \in \mathbb{R}\end{equation}
    \item \begin{equation}\lim_{x \to a} x = a\end{equation}
    \item \begin{equation}\lim_{x \to a} [cf(x)] = c\lim_{x \to a} f(x)\ ,\ c \in \mathbb{R}\end{equation}
    \item \begin{equation}\lim_{x \to a} [f(x) \pm g(x)] = \lim_{x \to a} f(x) \pm \lim_{x \to a} g(x)\end{equation}
    \item \begin{equation}\lim_{x \to a} [f(x) \times g(x)] = \lim_{x \to a} f(x) \times \lim_{x \to a} g(x)\end{equation}
    \item \begin{equation}\lim_{x \to a} \left(\frac{f(x)}{g(x)}\right) = \frac{\lim_{x \to a} f(x)}{\lim_{x \to a} g(x)}\text{, if }\lim_{x \to a} g(x) \neq 0\end{equation}
    \item \begin{equation}\lim_{x \to a} [f(x)]^n = [\lim_{x \to a} f(x)]^n\end{equation}
\end{enumerate}

If $\lim_{x \to a} \left(\frac{f(x)}{g(x)}\right) \text{ and }\lim_{x \to a} g(x) = 0$ you must remove the factor of $x$ from $f(x)$ and $g(x)$ by simplifying to evaluate the limit. In most cases this requires forming a difference of squares on the numerator. If you cannot remove the factor of $x$, the limit does not exist. 

\subsection{Rate of Change}
The \boldindex{Rate of Change} of a function is how much the y-values of the function change relative to the x-values. 

\begin{enumerate}
    \item The \boldindex{average rate of change} of a function over a given interval $[a,b]$ is given by $\frac{f(b) - f(a)}{b - a}$. In other words, it is the slope of the secant that joins the points $(a,f(a))$ and $(b,f(b))$.

    \item The tangent to a function, $f$, at $x = a$, is the line that touches $f$ at $(a,f(a))$, and best approximates the function near $a$.
\end{enumerate}


The rate of change of a function at $x = a$ can be determined by evaluating the following limit: 
\begin{equation}\lim_ {h \to 0} \frac{f(a+h) - f(a)}{h}\end{equation}

This is often used to find and prove derivatives. 


\subsection{Derivatives}

The derivative of a function $f(x)$ (denoted as $f'(x)$) is a function such that $f'(x)$ at $x$ is equal to the slope of $f(x)$ at $x$. 

Use these rules to determine the derivative of a function. 

\begin{center}
    \begin{tabular}{|>{\centering\arraybackslash}m{6cm}|>{\centering\arraybackslash}m{6cm}|}
    \hline
    \multicolumn{2}{|c|}{\textbf{Basic Rules}} \\
    \hline
    \textbf{Function} & \textbf{Derivative} \\
    \hline
    $f(x) = c$ & $f'(x) = 0$ \\
    \hline
    $g(x) = c f(x)$ & $g'(x) = c f'(x)$ \\
    \hline
    $f(x) = x^n$ & $f'(x) = nx^{n-1}$ \\
    \hline
    \multicolumn{2}{|c|}{\textbf{Composite Rules}} \\
    \hline
    $h(x) = f(x) \pm g(x)$ & $h'(x) = f'(x) \pm g'(x)$ \\
    \hline
    $h(x) = f(x) g(x)$ & $h'(x) = f'(x) g(x) + f(x) g'(x)$ \\
    \hline
    $h(x) = \frac{f(x)}{g(x)}$ & $h'(x) = \frac{f'(x) g(x) - f(x) g'(x)}{g(x)^2}$ \\
    \hline
    $h(x) = f(g(x))$ & $h'(x) = f'(g(x)) g'(x)$ \\
    \hline
    \multicolumn{2}{|c|}{\textbf{Trigonometric Derivatives}} \\
    \hline
    $f(x) = \sin(x)$ & $f'(x) = \cos(x)$ \\
    \hline
    $f(x) = \cos(x)$ & $f'(x) = -\sin(x)$ \\
    \hline
    $f(x) = \tan(x)$ & $f'(x) = \sec^2(x)$ \\
    \hline
    $f(x) = \sec(x)$ & $f'(x) = \sec(x) \tan(x)$ \\
    \hline
    $f(x) = \cot(x)$ & $f'(x) = -\csc^2(x)$ \\
    \hline
    $f(x) = \csc(x)$ & $f'(x) = -\csc(x) \cot(x)$ \\
    \hline
    
    \multicolumn{2}{|c|}{\textbf{Exponential Derivatives}} \\
    \hline
    $f(x) = a^x$ & $f'(x) = \ln(a) a^x$ \\
    \hline

    \multicolumn{2}{|c|}{\textbf{Logarithmic Derivatives}} \\
    \hline
    $f(x) = \log_a(x)$ & $f'(x) = \frac{1}{\ln(a) x}$ \\
    \hline
    $f(x) = \ln(x)$ & $f'(x) = \frac{1}{x}$ \\
    \hline
    
    \end{tabular}
    \end{center}


\subsection{Implicit Differentiation}



\section{Vectors}


\pagebreak
\appendix
\section*{Equations List}
\addcontentsline{toc}{section}{Equations List}%


\pagebreak
\printindex

\pagebreak
\section*{Credits}
\addcontentsline{toc}{section}{Credits}%


\end{document}